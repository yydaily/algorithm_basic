\documentclass[border=5mm,12pt,tikz]{standalone}
\usetikzlibrary{3d,calc,backgrounds,patterns}
\usetikzlibrary{positioning}
\usepackage{tikz}
\usepackage{color}
\usepackage{fouriernc}
\usepackage{xeCJK}
\usepackage{amsmath}
\usepackage[colorlinks,linkcolor=blue]{hyperref}
\newcommand\base[3] {
	\node (#1) [draw, opacity=0.5, thick, text=black, minimum width=4cm, minimum height=2cm, align=center] at (#2, #3) {#1};
}
\newcommand\done[4] {
	\node (#1) [draw, opacity=0.5, thick, fill=green, fill opacity=0.7, text=black, minimum width=4cm, minimum height=2cm, align=center] at(#2, #3) {\href{#4}{#1}};
}
\newcommand\todo[3] {
	\node (#1) [draw, opacity=0.5, thick, fill=red, fill opacity=0.6, text=black, minimum width=4cm, minimum height=2cm, align=center] at(#2, #3) {#1};
}
\newcommand\arrow[2] {
	\draw [->, line width=1.2pt] (#1) -- (#2);
}
\begin{document}
\begin{tikzpicture}
	\node (base) [draw, opacity=0.5, thick, text=black, minimum width=4cm, minimum height=2cm, align=center] at (0,0) {基本的代码/\\ Debug能力};
	\node (comment) [above=of base,align=center] {白色方框表示一些汇总节点\\绿色方框表示已经完成的博客,可以直接点击\\红色方框表示尚未完成的博客,欢迎催更};
	\base{数学}{0}{-4};
	\base{数据结构}{-16}{-4};
	\done{树状数组}{-16}{-8}{https://yydaily.github.io/2021/10/11/algorithm/binary_indexed_tree/};
	\base{数论}{-8}{-8};
	\done{线性代数}{8}{-8}{https://yydaily.github.io/2021/10/18/algorithm/linear_algebra/};
	\done{矩阵}{8}{-12}{https://yydaily.github.io/2021/10/19/algorithm/matrix/};
	\todo{高斯消元法}{8}{-16};
	\base{图论}{24}{-4};
	\done{深度优先搜索}{24}{-8}{https://yydaily.github.io/2021/10/22/algorithm/dfs/};
	\done{线性同余方程}{-8}{-12}{https://yydaily.github.io/2021/10/15/algorithm/congruence_equation/};
	\done{费马小定理}{-8}{-16}{https://yydaily.github.io/2021/10/20/algorithm/fermat_little_theorem/};
	\done{逆元}{-8}{-20}{https://yydaily.github.io/2021/10/20/algorithm/inverse/};
	\done{中国剩余定理}{-8}{-24}{https://yydaily.github.io/2021/10/12/algorithm/chinese_remainder_theorem/};
	\done{欧几里得算法}{-16}{-12}{https://yydaily.github.io/2021/10/12/algorithm/gcd/};
	\done{扩展欧几里得}{-16}{-16}{https://yydaily.github.io/2021/10/14/algorithm/exgcd/};
	\todo{连分数}{0}{-12};
	\todo{Pell方程}{0}{-16};
	\base{组合数学}{16}{-8};
	\done{排列组合}{16}{-12}{https://yydaily.github.io/2021/10/24/algorithm/permutation_combination/};
	\done{容斥原理}{16}{-16}{https://yydaily.github.io/2021/10/26/algorithm/inclusion_exclusion/};
	\arrow{base}{数学};
	\arrow{base}{数据结构};
	\arrow{base}{图论};
	\arrow{数学}{数论};
	\arrow{数论}{欧几里得算法};
	\arrow{数论}{线性同余方程};
	\arrow{数论}{连分数};
	\arrow{欧几里得算法}{扩展欧几里得};
	\arrow{扩展欧几里得}{逆元};
	\arrow{线性同余方程}{费马小定理};
	\arrow{费马小定理}{逆元};
	\arrow{逆元}{中国剩余定理};
	\arrow{连分数}{Pell方程};
	\arrow{线性代数}{矩阵};
	\arrow{矩阵}{高斯消元法};
	\arrow{图论}{深度优先搜索};
	\arrow{数学}{线性代数};
	\arrow{数据结构}{树状数组};
	\arrow{数学}{组合数学};
	\arrow{组合数学}{排列组合};
	\arrow{排列组合}{容斥原理};
\end{tikzpicture}
\end{document}
