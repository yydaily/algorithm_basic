\newcommand\NumberTheory {
	\base{number_theory}{below left=3cm and 5cm of math}{数论};
	\arrow{math}{number_theory};
	\done{gcd}{below left=3cm and 5cm of number_theory}{\href{https://yydaily.github.io/2021/10/12/algorithm/gcd/}{欧几里得算法}};
	\arrow{number_theory}{gcd};
	\done{exgcd}{below=3cm of gcd}{\href{https://yydaily.github.io/2021/10/14/algorithm/exgcd/}{扩展欧几里得}};
	\arrow{gcd}{exgcd};
	\done{congruence_equation}{below=3cm of number_theory}{\href{https://yydaily.github.io/2021/10/15/algorithm/congruence_equation/}{线性同余方程}};
	\arrow{number_theory}{congruence_equation};
	\todo{multiply_inverse}{below=6cm of congruence_equation}{逆元};
	\arrow{congruence_equation}{multiply_inverse};
	\arrow{exgcd}{multiply_inverse};
	\done{chinese_remainder_theorem}{below=3cm of multiply_inverse}{\href{https://yydaily.github.io/2021/10/12/algorithm/chinese_remainder_theorem/}{中国剩余定理}};
	\arrow{multiply_inverse}{chinese_remainder_theorem};
	\todo{continued_fraction}{below right=3cm and 5cm of number_theory}{连分数};
	\arrow{number_theory}{continued_fraction};
	\todo{pell_function}{below=3cm of continued_fraction}{Pell方程};
	\arrow{continued_fraction}{pell_function};
}
\newcommand\LinearAlgebra {
	\done{linear_algebra}{below right=3cm and 5cm of math}{\href{https://yydaily.github.io/2021/10/18/algorithm/linear_algebra/}{线性代数}};
	\arrow{math}{linear_algebra};
	\todo{matrix}{below=3cm of linear_algebra}{矩阵};
	\arrow{linear_algebra}{matrix};
	\todo{gaussian_elimination}{below=3cm of matrix}{高斯消元法};
	\arrow{matrix}{gaussian_elimination};
}
\newcommand\Math {
	\base{math}{below right=3cm and 5cm of base}{数学};
	\arrow{base}{math};

	\NumberTheory
	\LinearAlgebra
}
